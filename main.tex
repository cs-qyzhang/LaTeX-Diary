\documentclass{diary}

\author{周树人}
\title{鲁迅日记}

% 指定起止日期,年份为可选参数
\BeginAt[1913]{2}{8}
\EndAt{11}{23}

% 指定字体。这里使用的为方正鲁迅行书,若要使用可前往
% https://www.foundertype.com/index.php/FontInfo/index/id/4682
% 购买
\setCJKmainfont{FZLuXunXingShuS}
\setmainfont{FZLuXunXingShuS}

% 以下两个命令是可选的,用于指定标题页的中文和英文字体,若
% 不指定则使用正文字体
\TitleCJKFont{FZLuXunXingShuS}
\TitleFont{FZLuXunXingShuS}

% 示例节选的鲁迅日记摘自 http://www.eywedu.org/luxunquanji/30-.htm
% 示例中的地点、日期和天气并不一定准确
% 由于所选字体中没有个别生僻字的字形,缺失的字体已找近似代替
\begin{document}

% 模板的封面。默认的封面比较简单,可自行修改\maketitle命令定制
\maketitle

% 改变地点,省份(或国家)为可选参数
\MoveTo[中国]{北京}

% 开始一篇新日记。前两个参数指定了月份和日期,第三个参数为
% 可选参数,表示 \textbf{改变} 天气,若不指定则使用之前的天气。
% 第三个参数也可以用表情来代表今日的心情。天气和心情图标对应
% 的命令详见 icon-name.html
%
% \textbf{注意}: 第三个参数要用中括号[]
% \textbf{注意}:该命令后面比需要有%,且命令与日记内容之间
%                不能有空行,否则会引入不必要的空格和空行
\Day{2}{8}[\windy]%
八日晴,风。上午赴部,车夫误辗地上所置橡皮水管,有似巡警者
及常服者三数人突来乱击之,季世人性都如野狗,可叹!午后赴留
黎厂买得朱长文《墨池编》一部六册,附朱象贤《印典》二册,
十元。又《陶庵梦忆》一部四册,一元,此为王文诰所编,刻于桂林,
虽单行本,然疑与《粤雅堂丛书》本同也。下午往看季市,则惘惘如
欲睡,即出。晚谷青来,假去二十元。

\MoveTo{兖州}%

\Day{6}{20}[\sunny]%
二十日晴。上午十点二十分发天津。车过黄河涯,有孺子十余人拾石
击人,中一客之额,血大出,众哗论逾时。夜抵兖州,有垂辫之兵时来
窥窗,又有四五人登车,或四顾,或无端促卧人起,有一人则提予网篮
而衡之,旋去。

\MoveTo{北京}%

\Day{9}{28}%
二十八日星期休息。又云是孔子生日也。昨汪总长令部员往国子监,
且须跪拜,众已哗然。晨七时往视之,则至者仅三四十人,或跪或立,
或旁立而笑,钱念扣又从旁大声而骂,顷刻间便草率了事,真一笑话。
闻此举由夏穗卿主动,阴鸷可畏也。归途过齐寿山家小坐。路遇张协
和方自季市寓出,复邀之同往,至午归。下午小睡。晚国子监送来
牛肉一方。紫佩来,即去。

\Day{11}{23}[\rainy]%
二十二日大雨,遂不赴部。晚饮于陈公猛家,为蔡孑民饯别也,
此外为蔡谷青、俞英崖、王叔眉、季市及余,肴膳皆素。夜作
均言三章,哀范君也,录存于此:风雨飘摇日,余怀范爱农。
华颠萎寥落,白眼看鸡虫。

世味秋荼苦,人间直道穷。奈何三月别,竟尔失畸躬!

海草国门碧,多年老异乡。狐狸方去穴,桃偶已登场。

故里寒云恶,炎天凛夜长。独沈清泠水,能否涤愁肠?

把酒论当世,先生小酒人。大圜犹茗汀,微醉自沈沦。

此别成终古,从兹绝绪言。故人云散尽,我亦等轻尘!

\end{document}